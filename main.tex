% Classic Beamer setup
\documentclass[aspectratio=169]{beamer}
\usetheme{default}
\usecolortheme{default}
\useinnertheme{default}
\useoutertheme{default}

% Grid background
\usepackage{tikz}
\usebackgroundtemplate{
  \tikz[overlay,remember picture]
    \draw[step=0.5cm,gray!30] (current page.south west) grid (current page.north east);
}

% Colors
\definecolor{myaccent}{HTML}{007ACC}
\setbeamercolor{title}{bg=gray!60, fg=black}
\setbeamercolor{frametitle}{bg=gray!60, fg=black}
\setbeamercolor{block title}{bg=black, fg=white}
\setbeamercolor{block body}{bg=white, fg=black}
\setbeamercolor{structure}{fg=black}
\setbeamertemplate{navigation symbols}{}

% Fonts
\usepackage{fontspec}
\usepackage{unicode-math}
\setmainfont[
  Path = fonts/,
  Extension = .ttf,
  UprightFont = UniversRegular,
  BoldFont = UniversBold
]{UniversRegular}
\setsansfont[
  Path = fonts/,
  Extension = .ttf,
  UprightFont = UniversRegular,
  BoldFont = UniversBold
]{UniversRegular}
\setmathfont{Latin Modern Math}
\newfontfamily\thesisserif{Latin Modern Roman}

% Fix \mathcal & \mathfrak
\usepackage{amsmath, amssymb}
\let\mathcal\relax
\DeclareMathAlphabet{\mathcal}{OMS}{cmsy}{m}{n}
\let\mathfrak\relax
\DeclareMathAlphabet{\mathfrak}{U}{euf}{m}{n}

% Extra packages
\usepackage{appendixnumberbeamer}
\usepackage{stmaryrd}
\usepackage{stackengine}
\usepackage{graphicx}
\usepackage{amsthm}
\usepackage{hyperref}
\usepackage{mdframed}

% Title info
\setbeamertemplate{headline}{}
\title{\Huge \textbf{Algebraic Logic of Secrets}}
\author{Made By: \textbf{Omar Gamal Eldin}\\Supervised by: \textbf{Prof. Haythem O. Ismail}}
\institute{}
\date{}

% GUC logo
\addtobeamertemplate{title page}{
  \begin{tikzpicture}[remember picture,overlay]
    \node[anchor=south west,xshift=1.15em,yshift=1.1em] at (current page.south west) {
      \includegraphics[height=1.4cm]{images/GUC.jpg}
    };
  \end{tikzpicture}
}{}

% Outline bullets
\setbeamertemplate{section in toc}{\textbullet\ \inserttocsection}
\setbeamertemplate{subsection in toc}{\hspace{1.5em}\textbullet~\inserttocsubsection\\}

% Frametitle font size
% === Frametitle font size ===
\setbeamerfont{frametitle}{size=\Huge,series=\bfseries}

% Custom commands

\newcommand{\nec}{\Box}
\newcommand{\pos}{\Diamond}
\newcommand{\BoxStar}{\fboxsep=1pt\fbox{$\ast$}}

\begin{document}

\frame{\titlepage}

%=========Outline=========%

\begin{frame}
\frametitle{Outline}
% \small
\tableofcontents[hideallsubsections]
\end{frame}

%=========Introduction=========%

\section{Introduction}

\begin{frame}
\frametitle{Introduction}
\begin{itemize}
    \item Artificial agents are increasingly embedded in human environments.
    \item Understanding social concepts like secrecy is essential for trust and privacy.
    \item Existing logic frameworks often ignore everyday secrecy scenarios.
    \item This work bridges that gap with a new formal logic of secrets.
\end{itemize}
\end{frame}

\subsection{Motivation \& Aim}

\begin{frame}
\frametitle{Motivation \& Aim}
\begin{itemize}
  \item Agents must recognize, respect, and act on secrets in real-world settings.
  \item Existing models focus on security protocols, not social interactions.
  \item This thesis develops a logic that combines:
  \begin{itemize}
    \item Belief
    \item Intention
    \item Revelation
    \item Time
  \end{itemize}
  \item Aim: A compositional, algebraic logic for secrecy in multi-agent systems.
\end{itemize}
\end{frame}

%=========Background=========%

\section{Background}
\begin{frame}
\frametitle{Propositional \& Modal Logic}
\subsection{Propositional \& Modal Logic}
\begin{itemize}
    \item Propositional Logic (PPL):
    \begin{itemize}
        \item Basic true/false statements combined with connectives.
        \item No context or agents.
    \end{itemize}
    \item Propositional Modal Logic (PML):
    \begin{itemize}
        \item Adds modal operators:
        \[
        \Box p, \quad \Diamond p
        \]
        \item Uses possible worlds semantics.
    \end{itemize}
    \item Foundation for modeling knowledge, belief, and secrecy.
\end{itemize}
\end{frame}

\begin{frame}
\frametitle{Kripke Models}
\begin{itemize}
    \item Used to give semantics to PML and FOML.
    \item Structure:
    \begin{itemize}
        \item Possible worlds (states).
        \item Accessibility relation between worlds.
        \item Valuation function for propositions.
    \end{itemize}
    \item Notation:
    \[
    \mathcal{M = (W, R, A)}
    \]
    \item Limitations:
    \begin{itemize}
        \item Complex for real-world agents.
        \item Leads to paradoxes in belief and secrecy.
    \end{itemize}
    \item Algebraic logic avoids these by replacing worlds with algebraic operations.
\end{itemize}
\end{frame}


%=========Theory of Secrets=========%

\section{Theory of Secrets}

\subsection{Intuitions About Secrets}
\begin{frame}
\frametitle{Intuitions About Secrets}
\begin{itemize}
    \item Secrets play a key role in trust, privacy, and social reasoning.
    \item A secret is not an object but a relation between information and agents.
    \item Requires:
    \begin{itemize}
        \item Secret keepers (know and keep it)
        \item Nescients (do not know)
        \item A secrecy condition
    \end{itemize}
    \item Secrets are temporary and context-dependent.
\end{itemize}
\end{frame}

\subsection{Formal Structure of Secrets}
\begin{frame}
\frametitle{Formal Structure of Secrets}
\begin{itemize}
    \item Key components:
    \begin{itemize}
        \item Secretum ($\phi$): the confidential proposition.
        \item Secret Keepers (K): agents maintaining secrecy.
        \item Nescients (N): agents from whom the secret is hidden.
        \item Secrecy Condition ($\psi$): governs how long secrecy holds.
        \item Time (t): when secrecy is evaluated.
    \end{itemize}
    \item Notation: 
    \[
    Secret(\phi, K, N, \psi, t)
    \]
\end{itemize}
\end{frame}

\subsection{FOML Representation}
\begin{frame}
\frametitle{First-Order Modal Logic (FOML)}
\begin{itemize}
    \item Secrecy modeled using FOML with quantification and time.
    \item Core operators:
    \[
    B(A, \phi), \quad I(A, \phi), \quad R(A, \phi), 
    \quad H(\phi, t), \quad Mem(A, G)
    \]
    \item Short form: $O(\alpha, \beta, t)$ means $H(O(\alpha, \beta), t)$.
\end{itemize}
\end{frame}

\subsection{Types of Secrets}
\begin{frame}
\frametitle{Types of Secrets}
\begin{itemize}
    \item $Secret_0$: basic secrecy (keepers believe, intend, and do not believe it's revealed).
    \item Stronger types:
    \begin{itemize}
        \item $S_1$: $\phi$ not actually revealed.
        \item $S_2$: keepers know they keep it.
        \item $S_3$: keepers believe it hasn't been revealed.
        \item $S_4$: keepers know other keepers.
        \item $S_5$: keepers believe all co-keepers know secrecy.
    \end{itemize}
    \item Real secrets often satisfy multiple types.
\end{itemize}
\end{frame}

\subsection{Modeling Time with VEL}
\begin{frame}
\frametitle{Modeling Time with VEL}
\begin{itemize}
    \item Versatile Event Logic (VEL) structures time as branching histories.
    \item Each moment has:
    \begin{itemize}
        \item Unique past
        \item Multiple possible futures
    \end{itemize}
    \item Secrets persist unless an event changes them.
    \item Use:
    \[
    Occurs(e, \delta), \quad H(\phi, t)
    \]
\end{itemize}
\end{frame}

%=========Algebraic Logic Foundations=========%

\section{Algebraic Logic Foundations}

\subsection{Modal vs Algebraic Logic}
\begin{frame}
\frametitle{Modal vs Algebraic Logic}
\begin{itemize}
    \item Modal logic uses possible worlds to model knowledge and secrecy.
    \item Requires accessibility relations and world structures.
    \item Algebraic logic replaces worlds with compositional algebraic operations.
    \item Provides clearer, paradox-free reasoning.
    \item Better suited for explainable agent models.
\end{itemize}
\end{frame}

\subsection{Algebraic Logic of Belief ($Log_AB$)}
\begin{frame}
\frametitle{Algebraic Logic of Belief ($Log_AB$)}
\begin{itemize}
    \item Avoids possible-world semantics by using Boolean algebra.
    \item Belief is a function:
    \[
    B(a, p)
    \]
    where:
    \begin{itemize}
        \item $a$: agent
        \item $p$: proposition
    \end{itemize}
    \item Properties:
    \begin{itemize}
        \item Consistency
        \item Introspection
        \item Distribution
    \end{itemize}
    \item No paradoxes from self-reference.
\end{itemize}
\end{frame}

\begin{frame}
\frametitle{Properties of $Log_AB$}
\begin{itemize}
    \item LogAB models belief using Boolean algebra.
    \item Key properties:
    \begin{itemize}
        \item \textbf{Consistency:} Agents do not believe contradictions.
        \item \textbf{Positive Introspection:} Agents know what they believe.
        \item \textbf{Negative Introspection:} Agents know what they do not believe.
        \item \textbf{Distribution:} Belief distributes over logical connectives.
    \end{itemize}
    \item Example:
    \[
    B(a, p \wedge q) \equiv B(a, p) \wedge B(a, q)
    \]
    \item Avoids paradoxes by not relying on possible worlds.
\end{itemize}
\end{frame}

\subsection{Algebraic Logic of States ($Log_AS$)}
\begin{frame}
\frametitle{Algebraic Logic of States ($Log_AS$)}
\begin{itemize}
    \item Extends $Log_AB$ to model states over time.
    \item States are terms in a Boolean algebra.
    \item Uses:
    \[
    HoldsAt(s, t), \quad t_1 \prec t_2
    \]
    \item Classifies states:
    \begin{itemize}
        \item Eternal, Permanent, Temporary, etc.
    \end{itemize}
    \item Unifies states and propositions in one framework.
\end{itemize}
\end{frame}

%=========Algebraic Logic of Secrets=========%

\section{Algebraic Logic of Secrets ($Log_ASec$)}

\subsection{Language: Syntax and Semantics}
\begin{frame}
\frametitle{$Log_ASec$: Language}
\begin{itemize}
    \item Combines $Log_AB$, $Log_AS$, and VEL into one framework.
    \item Many-sorted algebraic structure:
    \begin{itemize}
        \item Propositions ($\sigma_P$)
        \item Agents ($\sigma_A$)
        \item Groups ($\sigma_G$)
        \item Time ($\sigma_T$)
        \item States ($\sigma_S$)
    \end{itemize}
    \item Adds:
    \[
    B(a, p), \; I(a, p), \; R(a, p), \;
    Mem(a, G), \; HoldsAt(p, t), \; t_1 \prec t_2
    \]
\end{itemize}
\end{frame}

\subsection{Axioms, Theorems, and Proof Sketches}
\begin{frame}
\frametitle{Key Axioms and Theorems}
\begin{itemize}
    \item Belief follows KD45:
    \begin{itemize}
        \item Consistency, introspection, distribution.
    \end{itemize}
    \item Intention follows KD:
    \begin{itemize}
        \item Consistent intentions.
    \end{itemize}
    \item Revelation:
    \begin{itemize}
        \item Based on evidence.
        \item Propagates truth.
    \end{itemize}
    \item Theorems:
    \[
    B(a, p) \Rightarrow R(a, p), \;
    R(a, p \wedge q) \Rightarrow R(a, p) \wedge R(a, q), \;
    R(a, p) \Leftrightarrow R(a, R(a, p))
    \]
\end{itemize}
\end{frame}

\begin{frame}
\frametitle{Proof Sketches}
\begin{itemize}
    \item Proofs follow directly from axioms.
    \item Example: $R(a, p \wedge q) \Rightarrow R(a, p)$ using R3.
    \item Belief-Revelation bridge:
    \begin{itemize}
        \item $B(a, p)$ + BR1 $\Rightarrow$ $R(a, p)$.
    \end{itemize}
    \item Logical omniscience affects proofs:
    \begin{itemize}
        \item Some fail if K axiom is removed.
    \end{itemize}
\end{itemize}
\end{frame}

\subsection{Removing Logical Omniscience}
\begin{frame}
\frametitle{Removing Logical Omniscience}
\begin{itemize}
    \item Classical logics assume agents know all consequences of beliefs.
    \item Unrealistic for bounded agents.
    \item $Log_ASec$ drops the K axiom:
    \[
    B(a, p) \wedge B(a, p \Rightarrow q) \Rightarrow B(a, q)
    \]
    \item Result:
    \begin{itemize}
        \item No forced deduction.
        \item Some theorems fail, e.g., $B(a, p) \Rightarrow B(a, R(a, p))$.
    \end{itemize}
    \item Models resource-limited reasoning.
\end{itemize}
\end{frame}

%=========Example=========%

%=========EXAMPLE SCENARIO=========%

\section{Example Scenario}

\begin{frame}
\frametitle{Example: Secret in a Group}
\begin{itemize}
    \item Scenario:
    \begin{itemize}
        \item Agents A, B, and C.
        \item Secret $\phi$ known by A and B.
        \item C must not know $\phi$.
    \end{itemize}
    \item Formalization:
    \[
    Secret(\phi, \{A,B\}, \{C\}, \psi, t)
    \]
    \item Secrecy condition $\psi$: Holds until A or B reveals $\phi$.
\end{itemize}
\end{frame}

\begin{frame}
\frametitle{How $Log_ASec$ Models It}
\begin{itemize}
    \item Belief:
    \[
    B(A, \phi), \; B(B, \phi), \; \neg B(C, \phi)
    \]
    \item Intention:
    \[
    I(A, \neg R(A, \phi)), \; I(B, \neg R(B, \phi))
    \]
    \item Membership:
    \[
    Mem(A, K), \; Mem(B, K), \; Mem(C, N)
    \]
    \item Temporal:
    \[
    HoldsAt(\phi, t), \; HoldsAt(\psi, t)
    \]
\end{itemize}
\end{frame}

\begin{frame}
\frametitle{Secrecy Evolution}
\begin{itemize}
    \item If A reveals $\phi$ at $(h,t)$:
    \begin{itemize}
        \item $R(A, \phi)$ holds at $(h,t)$.
        \item Secrecy condition $\psi$ no longer holds at $(h,t)$.
        \item C may learn $\phi$ in the future.
    \end{itemize}
    \item $Log_ASec$ tracks this via:
    \[
    \llbracket R(A, \phi) \rrbracket^{h,t} = \text{true}
    \Longrightarrow 
    \llbracket \psi \rrbracket^{h,t} = \text{false}
    \]
    \item Supports reasoning about possible future histories.
\end{itemize}
\end{frame}

%=========Conclusion & Future Work=========%

\section{Conclusion and Future Work}

\subsection{Key Contributions}
\begin{frame}
\frametitle{Key Contributions}
\begin{itemize}
    \item Developed a new algebraic logic for secrecy.
    \item Unified belief, intention, revelation, and time in one framework.
    \item Avoided paradoxes of possible-worlds semantics.
    \item Modeled secrecy evolution over branching time.
    \item Provided formal proofs and an example scenario.
\end{itemize}
\end{frame}

\subsection{Future Directions}
\begin{frame}
\frametitle{Future Directions}
\begin{itemize}
    \item Extend $Log_ASec$ to handle uncertainty and noisy channels.
    \item Integrate with AI planning and multi-agent systems.
    \item Automate secrecy checks using theorem provers.
    \item Apply to real-world privacy and information security use cases.
    \item Could be implemented in Prolog for automated secrecy reasoning.
\end{itemize}
\end{frame}

%=========References=========%

\section{References}
\begin{frame}
\frametitle{References}
\footnotesize
\begin{itemize}
    \item [1] Brandon Bennett and Antony P. Galton. A unifying semantics for time and events.
    Artificial Intelligence, 153(1):13–48, 2004. Logical Formalizations and Commonsense
    Reasoning.
    \item [2] Patrick Blackburn, Maarten de Rijke, and Yde Venema. Modal Logic. Cambridge
    University Press, 2001.
    \item [3] Thomas Bolander. Self-Reference and Paradox. In Edward N. Zalta and Uri Nodel-
    man, editors, The Stanford Encyclopedia of Philosophy. Metaphysics Research Lab,
    Stanford University, Fall 2024 edition, 2024.
    \item [4] Ronald Fagin, Joseph Y. Halpern, Yoram Moses, and Moshe Y. Vardi. Reasoning
    About Knowledge. MIT Press, 1995.
    \item [5] James Garson. Modal Logic. In Edward N. Zalta and Uri Nodelman, editors, The
    Stanford Encyclopedia of Philosophy. Metaphysics Research Lab, Stanford Univer-
    sity, Spring 2024 edition, 2024.
    \item [6] Paul R. Halmos. The basic concepts of algebraic logic. Journal of Symbolic Logic,
    23(2):223–223, 1958.
    \item [7] Joseph Y. Halpern and Kevin R. O’Neill. On the logic of secrecy. Journal of Com-
    puter Security, 12(2):229–272, 2004.
\end{itemize}
\end{frame}

\begin{frame}
\frametitle{References (cont.)}
\footnotesize
\begin{itemize}
    \item [8] Peter Hawke, Ayb¨uke ¨Ozg¨un, and Francesco Berto. The fundamental problem of
    logical omniscience. Journal of Philosophical Logic, 49(4):727–766, 2020.
    \item [9] Haythem O. Ismail. LogAB: A first-order, non-paradoxical, algebraic logic of belief.
    Logic Journal of the IGPL, 20(5):774–795, 03 2012.
    \item [10] Haythem O. Ismail. Stability in a commonsense ontology of states. Proceedings of
    the Eleventh International Symposium on Logical Formalization of Commonsense
    Reasoning (COMMONSENSE 2013), Agya Napa, Cyprus, 2013.
    \item [11] Haythem O. Ismail and Merna Shafie. A commonsense theory of secrets. IOS Press,
    Formal Ontology in Information Systems:77–91, 2020.
    \item [12] Robert Kowalski and Marek Sergot. A logic-based calculus of events. New Generation
    Computing, 4(1):67–95, 1986.
    \item [13] John-Jules Meyer and Wiebe van der Hoek. Epistemic Logic for AI and Computer
    Science. Cambridge University Press, 2004.
\end{itemize}
\end{frame}
  
%=========End=========%

\begin{frame}[c]
\centering
\Huge
\textbf{Thank you!}

\vspace{1cm}

\LARGE
Questions?
\end{frame}

%========= =========%

\end{document}